\begin{abstract}

    \begin{center}
    \textbf{\large{Применение мультиагентного обучения с подкреплением для динамических игр двух лиц}} \\ 
    \large{Гимишян Ашот} \\ [1 cm]
    \end{center}


\begin{center}
\begin{minipage}{0.8\textwidth}

Работа посвящена исследованию применения мультиагентного обучения с подкреплением в динамических играх двух лиц. В ней представлена модифицированная задача о сделке --- экспериментальная игра из теории игр. Данная игра может быть рассмотрена как модель многих реальных ситуаций, где стороны ведут переговоры о разделе общего ресурса. Она иллюстрирует сложность подобных переговоров и важность учета различных факторов, включая текущую и будущую выгоду, риск, а также влияние на соперника. В ходе исследования была разработана модель искусственного интеллекта, которая путем симуляции двусторонних переговоров эффективно решает данную задачу. Проведенные эксперименты подтвердили практическую значимость модели, отраженную в монотонном возрастании средней награды и уровня согласия между участниками. Полученные результаты могут быть применены для моделирования переговоров и достижения взаимоприемлемых результатов.\\
\end{minipage}
\end{center}

\vspace{20mm}
\textbf{Ключевые слова:} Искусственный интеллект · Машинное обучение · Нейронные сети · Обучение с подкреплением · Теория игр · Динамические игры ·  Модифицированная задача о сделке · Переговоры


    \vfill % используется для заполнения вертикального пространства в документе до конца страницы
    
\end{abstract}